\documentclass[10pt]{article}

\usepackage{booktabs}
\usepackage{hyperref}
\usepackage[most]{tcolorbox}

\newtcblisting{cisco}[1][]{size=fbox, listing only, listing options={style=tcblatex,basicstyle=\ttfamily\scriptsize,tabsize=2,language=sh},#1}

\renewcommand{\arraystretch}{1.2}
\renewcommand\thesubsubsection{\thesubsection.\alph{subsubsection}}

\begin{document}

\title{Cisco Certified Network Associate (200-125)}
\author{Joshua French}
\date{\vspace{-5ex}}

\maketitle

\renewcommand{\abstractname}{Exam Description}
\begin{abstract}
The Cisco Certified Network Associate (CCNA) Routing and Switching composite
exam (200-125) is a 90-minute, 50–60 question assessment that is associated with the CCNA Routing
and Switching certification. This exam tests a candidate's knowledge and skills related to network
fundamentals, LAN switching technologies, IPv4 and IPv6 routing technologies, WAN technologies,
infrastructure services, infrastructure security, and infrastructure management.

The following topics are general guidelines for the content likely to be included on the exam. However,
other related topics may also appear on any specific delivery of the exam. In order to better reflect the
contents of the exam and for clarity purposes, the guidelines below may change at any time without
notice.
\end{abstract}

\tableofcontents

\newpage
\section{Network Fundamentals}

\subsection{Compare and contrast OSI and TCP/IP models}
\begin{tabular}{@{}lll@{}}
\toprule
& OSI & TCP/IP \\ \midrule
Layers & 7 Layers & 4 Layers \\
Type & General-purpose model & Only applies to TCP/IP \\
\bottomrule
\end{tabular}

\subsection{Compare and contrast TCP and UDP protocols}
\begin{tabular}{@{}ll@{}}
\toprule
TCP & UDP \\ \midrule
Connection-oriented & Connectionless \\
\bottomrule
\end{tabular}

\subsection{Describe the impact of infrastructure components in an enterprise network}
\subsubsection{Firewalls}
Firewalls restrict movement of packets between sections of the network. They can be independent devices or integrated into hosts or routers.

\subsubsection{Access points}
Access points bridge wireless networks to wired networks. They can be used with or without a controller.
They utilize some wireless standard (usually something 802.11) to communicate with clients and bridge that traffic to either a wired network or 
another wireless network (wireless backhaul or mesh networking).

\subsubsection{Wireless controllers}
Wireless controllers control access points and can provide a path for traffic from wireless clients to the rest of the network. They usually are used 
with lightweight access points that cannot be configured on their own.

\subsection{Describe the effects of cloud resources on enterprise network architecture}
\subsubsection{Traffic path to internal and external cloud services}
Cloudy cloud cloud
\subsubsection{Virtual services}
Cloudy cloud cloud
\subsubsection{Basic virtual network infrastructure}
VXLAN?

\subsection{Compare and contrast collapsed core and three-tier architectures}
Table here
\subsection{Compare and contrast network topologies}
Table here
\subsubsection{Star}
List here
\subsubsection{Mesh}
List here
\subsubsection{Hybrid}
List here

\subsection{Select the appropriate cabling type based on implementation requirements}
Fiber vs copper?

\subsection{Apply troubleshooting methodologies to resolve problems}
\subsubsection{Perform and document fault isolation}
How to

\subsubsection{Resolve or escalate}
How to

\subsubsection{Verify and monitor resolution}
How to

\subsection{Configure, verify, and troubleshoot IPv4 addressing and subnetting}
\begin{cisco}[title=Set IP Address]
conf t
int g0/0
ip address 172.16.48.1 255.255.255.0
no shut
exit
\end{cisco}

\subsection{Compare and contrast IPv4 address types}
\subsubsection{Unicast}
Unicast addresses refer to a single host.

\subsubsection{Broadcast}
Broadcast addresses refer to hosts within the broadcast scope.

\subsubsection{Multicast}
Multicast addresses refer to all hosts subscribed to a multicast address.

\subsection{Describe the need for private IPv4 addressing}
Fake news.

\subsection{Identify the appropriate IPv6 addressing scheme to satisfy addressing requirements in a LAN/WAN environment}
/32 for the company
/48 for each location
/56 for each building
/64 for each vlan

\subsection{Configure, verify, and troubleshoot IPv6 addressing}
conf t
int g0/0
ipv6 address fd00::1/64
no shut
end

\subsection{Configure and verify IPv6 Stateless Address Auto Configuration}
conf t
int g0/0
ipv6 address slaac
no shut
end

\subsection{ Compare and contrast IPv6 address types}
\subsubsection{Global unicast}
This is a test
\subsubsection{Unique local}
ULA addresses

\subsubsection{Link local}
Addresses only valid on the link.

\subsubsection{Multicast}
Addresses

\subsubsection{Modified EUI 64}
Soooo many types

\subsubsection{Autoconfiguration}
Wut?

\subsubsection{Anycast}
Awwww yeah

\section{LAN Switching Technologies}
This is a test

\listoftables
\end{document}
